\documentclass[11pt]{article}
\usepackage{fullpage}
\usepackage{amsmath,amsfonts,amsthm,amssymb}
\usepackage{url}
\usepackage[demo]{graphicx}
\usepackage{caption} 
\usepackage{algpseudocode}
\usepackage{bbm}
\usepackage{float}
\usepackage{framed}
\usepackage{enumerate}
\usepackage{color}
\usepackage[colorlinks=true, linkcolor=red, urlcolor=blue, citecolor=blue]{hyperref}

\DeclareMathOperator*{\E}{\mathbb{E}}
\let\Pr\relax
\DeclareMathOperator*{\Pr}{\mathbb{P}}
\DeclareMathOperator*{\R}{\mathbb{E}}

\topmargin 0pt
\advance \topmargin by -\headheight
\advance \topmargin by -\headsep
\textheight 8.9in
\oddsidemargin 0pt
\evensidemargin \oddsidemargin
\marginparwidth 0.5in
\textwidth 6.5in

\parindent 0in
\parskip 1.5ex

\newcommand{\homework}[2]{
	\noindent
	\begin{center}
		\framebox{
			\vbox{
				\hbox to 6.50in { {\bf NYU CS-GY 6923: Machine Learning} \hfill Fall 2024 }
				\vspace{4mm}
				\hbox to 6.50in { {\Large \hfill Homework #1  \hfill} }
				\vspace{2mm}
				\hbox to 6.50in { {Name: #2 \hfill} }
			}
		}
	\end{center}
	\vspace*{4mm}
}

\begin{document}
	
	\homework{1}{Star Student}
	
	\section*{Problem 1}
	\textbf{Collaborators:} List all collaborators here, or simply write ``None'' if you worked alone.
	\medskip
	
	Problems will be answered with a combination of written explanations, equations, plots, and images. 
	
	\begin{align*}
	\E[X + Y]  &= \E[X] + \E[Y] \\
	& = 42.
	\end{align*}
	
	If you reference any outside material or theorems not discussed in class, please add a link and explanation: \href{https://xkcd.com/2042/}{https://xkcd.com/2042/}.
	
	\begin{figure}[h]
		\centering
		\includegraphics[width=0.3\textwidth]{image.jpg}
		\caption{Please explain all plots or diagrams with a sufficiently detailed caption!}
	\end{figure}
	
	Occasionally you may be asked to include code snippets:
	\begin{verbatim}
	def majorityElement(self, nums):
	count = 0
	candidate = None
	
	for num in nums:
	if count == 0:
	candidate = num
	count += (1 if num == candidate else -1)
	
	return candidate
	\end{verbatim}
	
	And sometimes it might be helpful to include pseudocode:
	
	\begin{algorithmic}
		\State $count \gets 0$
		\State $candidate \gets null$
		
		\For {$i \in {a_1,\dots,a_n}$}
		\If {$count == 0$}
		\State $candidate \gets a_i$
		\EndIf
		\State {$count \gets count + 1$}
		\EndFor	
		\State \Return $candidate$
	\end{algorithmic}
	
	
	\newpage
	\section*{Problem 2}
	\textbf{Collaborators:} List all collaborators here, or simply write ``None'' if you worked alone.
	\medskip
	
	Please start each new problem on a new page. This makes it easier to separate out problem sets during grading.
	
	
\end{document}